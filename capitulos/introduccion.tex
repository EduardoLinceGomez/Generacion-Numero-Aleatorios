\documentclass[../Main.tex]{subfiles}

\begin{document}

\chapteropening{C}{asi} todos los fenómenos observados en nuestra vida cotidiana o en la investigación científica tienen aspectos dinámicos importantes. Los ejemplos específicos pueden surgir en (a) un sistema físico, tal como un vehículo que viaja en el espacio, un sistema de calefacción de casa o en la explotación de un yacimiento minero; (b) un sistema social, tales como el movimiento dentro de una jerarquía de una organización, la evolución de un sistema de clases tribales o el comportamiento de una estructura económica; o (c) un sistema de vida, tal como el de la transferencia genética, la decadencia ecológica o el crecimiento de la población.

Los números pseudoaleatorios son secuencias generadas por algoritmos deterministas que, pese a su origen, reproducen de manera controlada propiedades estadísticas de los números aleatorios. Una secuencia pseudoaleatoria se caracteriza por ser impredecible sin conocimiento previo y por superar pruebas estadísticas clásicas, aunque pueda derivarse de reglas simples \cite{Lehmer1951}. Su importancia radica en la amplia aplicabilidad que tienen en disciplinas como criptografía, simulación, métodos numéricos, biología, física, finanzas, inteligencia artificial y telecomunicaciones.

Los \textbf{generadores físicos} de números aleatorios, que utilizan fenómenos naturales como el decaimiento radioactivo, el ruido térmico en dispositivos electrónicos, la detección de rayos cósmicos o fenómenos mecánicos \cite{Madiot2022,Stipcevic2014}, representan la forma más cercana a la verdadera aleatoriedad. Aunque estos métodos proporcionan secuencias altamente impredecibles, su procesamiento lento, sesgos inherentes y la incapacidad de reproducir las secuencias generadas limitan su uso en aplicaciones que requieren alta eficiencia y reproducibilidad.

Con el avance de la computación en el siglo XX, los generadores de números pseudoaleatorios ganaron relevancia. Estos generadores, también conocidos como cuasi-aleatorios, son algoritmos deterministas que exhiben comportamientos caóticos, ofreciendo soluciones a las limitaciones de los generadores físicos y produciendo resultados estadísticamente indistinguibles de la verdadera aleatoriedad.

Uno de los métodos deterministas más antiguos\footnote{Tiene sus inicios en 1951, siendo el GLC una versión del generador propuesto por Lehmer \cite{Lehmer1951}.} y simples es el \textbf{generador congruencial lineal (GLC)} \cite{Lehmer1951,Knuth1997,BaltazarLarios2024}, definido por la fórmula recursiva
\[ X_{n+1} = (aX_n + c) \mod m. \]
En esta expresión $X_n$ es el $n$-ésimo número pseudoaleatorio de la secuencia, \(a\) es el multiplicador, \(c\) es el incremento, \(m\) es el módulo y el valor inicial $X_0$ es la semilla. Aunque los GLC son rápidos y fáciles de implementar, pueden tener ciclos cortos y patrones repetitivos si los parámetros no se eligen cuidadosamente. Investigaciones recientes han abordado estas limitaciones proponiendo el uso de factores dinámicos \cite{Alhomdy2015}.

Generadores más modernos, como el \textbf{Mersenne Twister}\footnote{Su nombre proviene del hecho de que la longitud del periodo corresponde a un número primo de Mersenne.}, desarrollado por Makoto Matsumoto y Takuji Nishimura en 1997 \cite{Matsumoto1998}, utilizan herramientas de la teoría de números para generar secuencias con un largo período de \(2^{19937}-1\) y alta calidad estadística. Este generador es ampliamente utilizado debido a su eficiencia y robustez.

En criptografía, los algoritmos de tipo hash, como el \textbf{SHA-1 (Secure Hash Algorithm 1)} \cite{Eastlake2001}, juegan un papel crucial en la seguridad de los datos, generando números pseudoaleatorios que son difíciles de predecir.

En este trabajo nos centraremos en el uso de \textbf{sistemas dinámicos discretos} para la generación de números pseudoaleatorios. Los sistemas dinámicos, especialmente los caóticos, han demostrado ser útiles en esta área debido a su comportamiento complejo y aparentemente aleatorio. Con los avances en el poder computacional y la teoría de sistemas dinámicos caóticos \cite{Behnia2011,Szczepanski2001}, la generación de números pseudoaleatorios a través de estos sistemas ha ganado relevancia.

\end{document}
